This section serves to test the efficacy and performance of the proposed
method. The experimental procedure was conducted using a benchmark dataset $D$
consisting of $|D| = 778$ laser scans obtained from a Sick range-scan sensor
mounted on a robotic wheel-chair.\footnote{The dataset is available at
\url{https://censi.science/pub/research/2007-plicp/laserazosSM3.log.gz}} For
each scan $D^d$, $d = 1,\dots,778$, the dataset reports one range scan of $360$
range measurements and the pose from which it was captured
$\bm{r}^d(x,y,\theta)$.  The same dataset was used to evaluate the performance
of IDC \cite{idc}, ICP, and MBICP in \cite{mbicp}, and that of PLICP and the
joint method PLICP$\circ$GPM during scan-matching experiments. In \cite{plicp}
the latter was found to be the best-performing among the five
correspondence-finding scan-matching methods. Therefore for purposes of
comparison against correspondence-finding scan-matching methods that may be
utilised in scan--to--map-scan matching, the experimental procedure is extended
to PLICP$\circ$GPM. This method shall be denoted hereafter by the acronym CSM.
In the same vein, for purposes of comparison against correspondenceless
scan-matching methods, the same experimental procedure is extended to the
Normal Distributions Transform (NDT) scan-matching method \cite{ndt1}.

The experimental setup is the following. The rays of each dataset instance
$D^d$ are first projected to the $x-y$ plane around $\bm{r}^d$. The
dataset's scans are not panoramic, therefore the remaining space is filled
with a semicircular arc that joins the scan's two extreme ends. Its radius is
set to the minimum range between the two extreme rays of $D^d$. Similar
fashions for closing-off the environment have been found equivalent with
respect to the performance of the tested methods. The resulting point-set is
regarded as the environment $\bm{W}^d$ in which the range sensor operates (e.g.
the environment of fig. \ref{fig:the_problem}). Then the map of the environment
$\bm{M}^d$ is set to be $\bm{W}^d$. In order to induce distortions in the map,
each coordinate of all points in $\bm{M}^d$ is perturbed by errors extracted
from a normal distribution $\mathcal{N}_{\bm{M}} \sim (0, \sigma_{\bm{M}}^2)$.
What is considered the sensor's actual pose $\bm{p}^d$ is generated randomly
within the polygon formed by $\bm{W}^d$. The range scan $\mathcal{S}_R^d$ that
is considered to be reported by the physical sensor is then computed by
locating the intersection points between $N_s$ rays emanating from $\bm{p}^d$
and the polygon formed by $\bm{W}^d$ across an angular field of view $\lambda =
2\pi$.  The initial pose estimate of the sensor $\hat{\bm{p}}^d$ is then
obtained by perturbing the components of $\bm{p}^d$ with quantities extracted
from uniformly distributed error distributions
$U_{xy}(-\overline{\delta}_{xy}, \overline{\delta}_{xy})$,
$U_{\theta}(-\overline{\delta}_{\theta}, \overline{\delta}_{\theta})$;
$\overline{\delta}_{xy}$, $\overline{\delta}_\theta$
$\in \mathbb{R}_{\geq 0}$.

In order to test for the performance of the proposed method, four
levels of noise acting on the range measurements of the real scan
$\mathcal{S}_R^d$ are tested. The range measurements are perturbed by zero-mean
normally-distributed noise with standard deviation
$\sigma_R \in \{0.01, 0.03, 0.05, 0.10\}$ m.\footnote{The values of tested
standard deviations were calculated from commercially available panoramic LIDAR
scanners by identifying the magnitude of their reported maximum range errors
and dividing it by a factor of three. The rationale is that $99.73\%$ of
errors are located within $3\sigma$ around the actual range between a ray and an
obstacle, assuming errors are distributed normally. The minimum standard
deviation $\sigma_R = 0.01$ m is reported for VELODYNE sensors
\cite{velodyne_datasheet}; the rest are reported for price-appealing but
disturbance-laden sensors, e.g. the RPLIDAR A2M8, or the YDLIDAR G4, TG30, and
X4 scanners \cite{a2m8_datasheet}-\cite{x4_datasheet}} In addition, two levels
of map distortion are tested: $\sigma_{\bm{M}} \in \{0.0, 0.05\}$ m.

For each experiment FSMSM, CSM, and NDT ran for $E = 100$ times across all
instances of $D$. CSM was set up as follows. GPM \cite{gpm} was used initially
in order to overcome the angular realignment problems \cite{plicp} of PLICP.
Then PLICP was executed for $I_{\text{PLICP}}$ times, and the total matching
error (eq. \ref{eq:s2sm_def}) was recorded. The pose estimate returned was
that which scored the lowest matching error across $I_{\text{PLICP}}$
iterations. GPM was called once because it was found to be impedimental to
convergence at medium to high levels of noise when called iteratively.
NDT was run for $I_{\text{NDT}}$ iterations. FSMSM's termination criterion was
set to CAER$(\hat{\bm{p}}^\prime) \leq (\hat{\sigma}_R + \hat{\sigma}_V)^{1/2}$,
where $\hat{\sigma}_R$ and $\hat{\sigma}_V$ are estimates of the standard
deviation of noise affecting the rays of $\mathcal{S}_R$ and $\mathcal{S}_V$
respectively. All experiments and algorithms were run serially, on a single
thread, on a machine with a CPU frequency of $4.0$ GHz.

The criterion on which the evaluation of all tests rests is the $2$-norm of the
total pose error---eq. (\ref{eq:pose_error_def}) for $\hat{\bm{p}} \rightarrow
\hat{\bm{p}}^\prime$, where $\hat{\bm{p}}^\prime$ is the output of each
algorithm tested. For every pose estimate $\hat{\bm{p}}^\prime_d$ outputted by
each algorithm, $d = 1,2,\dots,|D|$, its offset from the actual pose $\bm{p}^d$
is recorded in the form of the $2$-norm total error. The pose errors of one
simulation are then averaged. The pose error distributions reported below are
those of mean errors across $E$ simulations of the same configuration. The
unit of measurement of the total pose error is
$(\text{m}^2+\text{rad}^2)^{1/2}$, and it is omitted in the figures of the
following subsections for reasons of economy of space.

Figures \ref{fig:errors_sm0} and \ref{fig:errors_sm5} show the distribution of
the three methods' mean pose errors across $E$ experiments for maximal
displacements $\overline{\delta}_{xy} = 0.20$ m and $\overline{\delta}_\theta =
\pi / 4$ rad, when $\sigma_{\bm{M}} = 0.0$ m and $\sigma_{\bm{M}} = 0.05$ m
respectively. The value of $\overline{\delta}_{xy}$ was chosen as such from
reports on positional errors in real conditions \cite{gangpeng}. The value of
$\overline{\delta}_\theta$ was chosen as such in order to include orientation
errors at the initialisation stage of pose tracking. The size of the input real
scan was set to $N_s=360$ rays. The minimum and maximum oversampling rates
of FSMSM were set to $(\mu_{\min},\mu_{\max}) = (2^{\nu_{\min}},2^{\nu_{\max}})
= (2^2,2^5)$. The number of iterations of the translational component
at each map sampling degree $\nu$ was set at $I = 2\nu$. The number of
iterations of PLICP and NDT were set to $I_{\text{PLICP}} = I_{\text{NDT}}=
10$; higher values did not yield improved results.

\begin{figure}[]\centering
  \input{./figures/experiments/boxplot_errors_CSM_KUF_NDT_sm0.tex}
  \caption{\small Pose errors of FSMSM, CSM, and NDT for maximal uniform
           position displacements $\delta_{xy} \in U_{xy}(-0.20, +0.20)$ m and
           maximal uniform orientation displacements
           $\delta_\theta \in U_\theta (-\pi / 4, + \pi /4)$ rad for
           $\sigma_{\bm{M}} = 0.0$ m over $E = 100$ runs per noise level tested
           $\sigma_R$}
  \label{fig:errors_sm0}
\end{figure}


