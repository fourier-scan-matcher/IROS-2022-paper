The previous two sections describe two methods of how it is possible to (a)
reduce the error of the orientation estimate when the position estimate
coincides with the sensor's position, and (b) reduce the error of the position
estimate when the orientation estimate equals the sensor's orientation. In the
general case, however, no equality stands. What is more is that the problem is
coupled: the optimal orientation error cannot be attained in one step when the
position error is not zero, and the optimal position error cannot be attained
in one step when the orientation error is not zero. Therefore the first goal of
a method reducing both would be to first reduce the orientation error and then
reduce the location error. The second would be to iterate this process until
some termination condition is met. This method is described in the following.

Given an input pose estimate $\hat{\bm{p}}(\hat{x}, \hat{y}, \hat{\theta})$,
the real scan $\mathcal{S}_R$, and the map $\bm{M}$, the pose correction method
proposed (fig. \ref{fig:outer_system}) reduces the error of the pose estimate
by iteratively invoking the One-step Pose Correction process (fig.
\ref{fig:inner_system}) until a set of termination conditions is met. Denoting
the former by FSMSM, FSMSM starts off with an initial degree of sampling the
map $\nu$ = $\nu_{\min}$. The input pose estimate is processed by the One-step
Pose Correction process, and its output $\hat{\bm{p}}^\prime$ is examined with
regard to Recovery and Convergence conditions. If the resulting pose estimate
falls outside of the map $\bm{M}$ then a new pose estimate is generated from
the initially supplied pose estimate, and the process is reset. If no
significant pose estimate correction is observed
$\|\hat{\bm{p}}^\prime-\hat{\bm{p}}\|_2 < \varepsilon_{\delta p}$, then the
degree of map sampling $\nu$ is increased. Its increase serves as a means of
reducing the orientation and hence the position estimate error further.
Otherwise, the One-step Pose Correction process is reiterated until no
significant correction is observed. The process is iterated until a maximum
degree of map sampling is reached $\nu$ = $\nu_{\max}$, at which point FSMSM
terminates if a terminal condition is met. This terminal condition facilitates
the avoidance of local maxima. In the case where this condition is not met,
a new pose is generated, and the process is reset.



Given an input pose estimate $\hat{\bm{p}}(\hat{x}, \hat{y}, \hat{\theta})$,
the real scan $\mathcal{S}_R$, the map $\bm{M}$, and a sampling degree $\nu$,
the One-step Pose Correction system first calculates $2^\nu$ pose estimates
$\hat{\bm{P}}_{OC} = \{(\hat{x}, \hat{y}, \hat{\theta}_k)\}$,
$k = 0,\dots,2^\nu$$-$$1$. The Orientation Correction system utilises Algorithm
\ref{alg:algorithm_ufrcnu}. Its operation is denoted in fig.
\ref{fig:inner_system} by the operator OC$(\cdot)$.

Now, if the position of the input pose estimate coincided with the position of
the real sensor, the Percent Discrimination metric (eq. \ref{eq:pd}) would
suffice in serving as an accurate determinant of the pose estimate with the
least orientation error. In practice, however, the ranking provided by the
Percent Discrimination metric is confounded by the incoincidence of the two
positions.  In order to mitigate this effect, each pose estimate in
$\hat{\bm{P}}_{OC}$ is given over to the Position Correction system, where the
position of each pose estimate is displaced once ($I$=$1$), according to
Algorithm \ref{alg:algorithm_icte}. This operation, denoted by the operator
RPC$(\cdot)$ in fig. \ref{fig:inner_system}, produces the set
$\hat{\bm{P}}_{RPC} = \{(\hat{x}_k, \hat{y}_k, \hat{\theta}_k)\}$,
$|\hat{\bm{P}}_{RPC}| = 2^\nu$.  The purpose of this operation is for it to
provide an advance view of the next step of position correction: the less
rotationally misaligned a pose estimate is, the less it will diverge in terms of
orientation and hence position with respect to the sensor's actual pose once
inputted to the position correction system. This divergence is captured by the
Cumulative Absolute Error per Ray (CAER) metric:
\begin{align}
  \text{CAER}_k = & \sum\limits_{n=0}^{N_s-1} \Bigg| \mathcal{S}_R[n] - \mathcal{S}_V[n]\Big|_{(\hat{x}_k, \hat{y}_k, \hat{\theta}_k)} \Bigg|
\end{align}
where $k$ = $0,\dots,2^\nu$$-$$1$. The CAER metric encodes at the same time a degree
of alignment of position and orientation between its two input
scans.\footnote{By contrast, dropping the absolute value operator would provide
only for a position alignment metric.} By rehearsing the position correction of
each pose estimate in $\hat{\bm{P}}_{OC}$ and capturing the CAER for each of
its displaced pose estimates in $\hat{\bm{P}}_{RPC}$, it is possible to
establish a pose error rank between pose estimates in $\hat{\bm{P}}_{OC}$ and
simultaneously retain only one pose estimate for the next iteration of the
One-step Pose Correction method.\footnote{Alternatively, correcting the
position of $2^\nu$ pose estimates and feeding them back to the One-step Pose
Correction method would incur exponential costs in time of execution.} The pose
estimate $\hat{\bm{p}}_C \in \hat{\bm{P}}_{OC}$ that, when translated once,
records the minimum CAER among all similarly-treated pose estimates in
$\hat{\bm{P}}_{OC}$ is inputted to the Position Correction method proper. The
number of translation iterations $I$ it undergoes is an increasing function in
the degree of map sampling $\nu$.\footnote{The rationale of chaining the number
of translational iterations to the map sampling degree $\nu$ is the following.
Since the orientation error is inversely proportional to $\nu$, at low map
sampling rates, when the position estimate error is at its highest, if the
number of translational iterations was high then the position estimate would be
susceptible to divergence. Therefore the number of translational iterations is
kept low at initial stages so that a balance between decreasing position error
and position divergence is struck. At higher values of $\nu$, the orientation
estimate error decreases, and then divergence is bounded and/or met at higher
translational iteration values.  As the orientation estimate becomes ever more
accurate, the Position Correction system is let to iterate more times so that
further reduction of the position error be feasible.} The Position Correction
system produces $\hat{\bm{p}}^\prime$, which is then fed back to the
Orientation Correction system in the form of its new pose estimate
$\hat{\bm{p}} \leftarrow \hat{\bm{p}}^\prime$. In practice, the pose set
$\hat{\bm{P}}_{OC}$ is supplemented with one pose whose position is equal to
$\hat{\bm{p}}$ and whose orientation is equal to the orientation of
$\hat{\bm{p}}_C$ that produces the minimum CAER over time. This addition
introduces a form of memory to the system, which assists it in avoiding
divergence and which, therefore, benefits speed of execution.
