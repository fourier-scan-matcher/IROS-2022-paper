The previous two sections describe two methods of how it is possible to (a)
estimate the relative orientation between two panoramic 2D range scans when
both are captured from the same position but from different orientations, and
(b) estimate their relative location when both are captured from the same
sensor orientation but from different locations. In the general case, however,
no equality stands. The following analysis describes how these two methods are
combined in tandem in order to solve Problem \ref{prob:the_problem}.

Let the assumptions of Problem \ref{prob:the_problem} hold. Then denote the
point-set that is the result of the projection of range scan $\mathcal{S}_0$ to
the $x-y$ plane around an arbitrary but fixed pose $\bm{p}_0$ by $\bm{M}$. Then
the objective is to estimate the pose $\bm{p}_1$ from where $\mathcal{S}_1$ was
captured relative to $\bm{p}_0$ by way of registering $\mathcal{S}_1$ to map
$\bm{M}$. In the following, pose $\bm{p}_0$ is treated as the pose estimate of
$\bm{p}_1$.


Given an input pose estimate $\bm{p}_0(x_0, y_0, \theta_0)$, range scan
$\mathcal{S}_1$, the map $\bm{M}$, and a sampling degree $\nu$, the One-step
Pose Estimation system (fig. \ref{fig:fsm_inner}) first calculates $2^\nu$ pose
estimates of $\bm{p}_1$: $\bm{P}_{OC} = \{(x_0, y_0, \theta_0^k)\}$,
$k = 0,\dots,2^\nu$$-$$1$, according to the orientation estimation method
described in section \ref{subsec:method_orientation_correction}.

If the location of $\bm{p}_0$ coincided with the location of $\bm{p}_1$, the
Percent Discrimination metric (eq. (\ref{eq:pd})) would suffice in serving as an
accurate determinant of the orientation of $\bm{p}_1$.  In practice, however,
the ranking provided by the Percent Discrimination metric is confounded by the
incoincidence of the two locations. In order to mitigate this effect, each pose
estimate in $\bm{P}_{OC}$ is given over to the Position Estimation
system, where the position of each pose estimate is displaced once ($I$=$1$),
according to the method described in section
\ref{subsec:method_location_correction}. This operation produces the pose set
$\bm{P}_{RPC} = \{(x_0^k, y_0^k, \theta_0^k)\}$, $|\bm{P}_{RPC}| = 2^\nu$. The
purpose of this operation is for it to provide an advance view of the next step
of location estimation: the less rotationally misaligned a pose estimate of
$\bm{p}_1$ is, the less it will diverge in terms of orientation and hence
position with respect to $\bm{p}_1$ once inputted to the position estimation
system. This divergence is captured by the Cumulative Absolute Error per Ray
(CAER) metric:
\begin{align} \text{CAER}_k =
  & \sum\limits_{n=0}^{N_s-1} \Bigg| \mathcal{S}_1[n] -
  \mathcal{S}_V[n]\Big|_{(\hat{x}_k, \hat{y}_k, \hat{\theta}_k)} \Bigg|
\end{align}
where $\mathcal{S}_V$ is the map-scan captured from $(\hat{x}_k, \hat{y}_k,
\hat{\theta}_k)$, $k$ = $0,\dots,2^\nu$$-$$1$, within $\bm{M}$. The CAER metric
encodes at the same time a degree of alignment of position and orientation
between its two input scans. By rehearsing the position estimation of each pose
estimate in $\bm{P}_{OC}$ and capturing the CAER for each of its displaced pose
estimates in $\bm{P}_{RPC}$, it is possible to establish a pose error rank
between pose estimates in $\bm{P}_{OC}$ and simultaneously retain only one pose
estimate for the next iteration of the One-step Pose Estimation
method.\footnote{Alternatively, correcting the position of $2^\nu$ pose
estimates and feeding them back to the One-step Pose Estimation method would
incur exponential costs in time of execution.} The pose estimate $\bm{p}_C \in
\bm{P}_{OC}$ that, when translated once, records the minimum CAER among all
similarly-treated pose estimates in $\bm{P}_{OC}$ is inputted to the Position
Estimation method proper. The number of translation iterations $I$ it undergoes
is an increasing function in the degree of map sampling $\nu$.
%\footnote{The rationale of chaining the number of translational
%iterations to the map sampling degree $\nu$ is the following.  Since the
%orientation error is inversely proportional to $\nu$, at low map sampling
%rates, when the position estimate error is at its highest, if the number of
%translational iterations was high then the position estimate would be
%susceptible to divergence. Therefore the number of translational iterations is
%kept low at initial stages so that a balance between decreasing position error
%and position divergence is struck. At higher values of $\nu$, the orientation
%estimate error decreases, and then divergence is bounded and/or met at higher
%translational iteration values.  As the orientation estimate becomes ever more
%accurate, the Position Estimation system is let to iterate more times so that
%further reduction of the position error be feasible.}
The Position Estimation system produces $\bm{p}_0^\prime$, which is then fed
back to the Orientation Estimation system in the form of a new pose estimate
of $\bm{p}_1$: $\bm{p}_0 \leftarrow \bm{p}_0^\prime$. In practice, the pose set
$\bm{P}_{OC}$ is supplemented with one pose whose location component is equal
to $\bm{p}_0$ and whose orientation is equal to the orientation of $\bm{p}_C$
that produces the minimum CAER over time. This addition introduces a form of
memory to the system, which assists it in avoiding divergence and which,
therefore, benefits speed of execution.


Given pose $\bm{p}_0$, range scan $\mathcal{S}_1$, and the map $\bm{M}$, the
pose estimation method proposed iteratively invokes the One-step Pose
Estimation process until a set of termination conditions is met. Denoting the
former by FSM (Fourier Scan Matching), FSM starts off with an initial degree of
sampling the map $\nu$ = $\nu_{\min}$. The input pose estimate  $\bm{p}_0$ is
processed by the One-step Pose Estimation process, and its output
$\bm{p}_0^\prime$ is examined with regard to Recovery and Convergence
conditions. If the resulting pose estimate falls outside of the map $\bm{M}$
then a new pose estimate is generated from the initially supplied pose
estimate, and the process is reset.  If no significant pose estimate correction
is observed $\|\bm{p}_0^\prime-\bm{p}_0\|_2 < \varepsilon_{\delta p}$, then the
degree of map sampling $\nu$ is increased.  Its increase serves as a means of
reducing the orientation and hence the position estimate error further.
Otherwise, the One-step Pose Estimation process is iterated until a maximum
degree of map sampling is reached $\nu$ = $\nu_{\max}$, at which point FSM
terminates. Its output is $\bm{p}_0^\prime$, which is the pose estimate of
$\bm{p}_1$ in the frame of reference of $\bm{M}$. The roto-translation
$\hat{\bm{T}} = \bm{p}_0^\prime - \bm{p}_0$ is the estimate of the sensor's
true motion $\bm{T}$.
