The experimental procedure was conducted using a collection $D = \{D_k\}, k =
1,\dots,5$ of five heterogeneous benchmark datasets and sensor properties,
courtesy of the Department of Computer Science, University of Freiburg,
comprising a total of $|D| = 45402$ scan measurements \cite{datasets_link}.
For purposes of comparison against state-of-the-art scan-matching methods the
experimental procedure is extended to the Normal Distributions Transform (NDT)
scan-matching method \cite{Biber}, FastGICP \cite{Segal2009} and PLICP
\cite{Censi2008b}.  NDT, FastGICP, and PLICP belong to the \textit{established}
state-of-the-art methods of scan-matching
\cite{Koide2021,Xu2018a,Sobreira2019a,Pishehvari2019a,Qingshan2019a,Pham2021a}.
In addition, the experimental procedure is extended to FastVGICP
\cite{Pham2021a} and NDT-PSO \cite{Bouraine2020}.

The experimental setup is the following. The rays of each dataset instance
$D^d, d = 1,2,\dots,|D|$ are first projected to the $x-y$ plane around
$\bm{r}^d$.  The dataset's scans are not panoramic, therefore the remaining
space is filled with a semicircular arc that joins the scan's two extreme ends.
Alternative fashions for closing-off the environment have been found equivalent
with respect to the performance of the tested methods. The resulting point-set
is regarded as the environment $\bm{W}^d$ in which the range sensor operates.
Then the pose $\bm{p}_0^d$ from which $\mathcal{S}_0^d$ is captured is
generated randomly within the polygon formed by $\bm{W}^d$. The pose
$\bm{p}_1^d$ from which the sensor captured $\mathcal{S}_1$ is then obtained by
perturbing the components of $\bm{p}_0^d$ with quantities extracted from
uniformly distributed error distributions $U_{xy}(-\overline{\delta}_{xy},
\overline{\delta}_{xy})$, $U_{\theta}(-\overline{\delta}_{\theta},
\overline{\delta}_{\theta})$; $\overline{\delta}_{xy}$,
$\overline{\delta}_\theta$ $\in \mathbb{R}_{\geq 0}$.

Range scans $\mathcal{S}_0^d$ and $\mathcal{S}_1^d$ are then computed by
locating the intersection points between $N_s$ rays emanating from $\bm{p}_0^d$
and $\bm{p}_1^d$, respectively, and the polygon formed by $\bm{W}^d$ across an
angular field of view $\lambda = 2\pi$. The inputs to all algorithms are
then set to $\mathcal{S}_0^d$ and $\mathcal{S}_1^d$. Their output is
$\bm{p}_1^{\prime d}$. The roto-translation
$\hat{\bm{T}}^d = \bm{p}_1^{\prime d}$ is the estimate of the motion
$\bm{T}^d = \bm{p}_1^d - \bm{p}_0^d$ of the range sensor.

For every pose estimate $\bm{p}_1^{\prime d}$ outputted by each algorithm, its
offset from the actual pose $\bm{p}_1^d$ is recorded in the form of the
orientation error and the $2$-norm position error.  In order to test for the
performance of the proposed method with use of real sensors, five levels of
noise acting on the range measurements of the scans are tested. The range
measurements are perturbed by zero-mean normally-distributed noise with
standard deviation $\sigma_R \in \{0.01, 0.03, 0.05, 0.10, 0.20\}$ m.  The
values of tested standard deviations were calculated from commercially
available panoramic LIDAR scanners by identifying the magnitude of their
reported maximum range errors and dividing it by a factor of three. The
rationale is that $99.73\%$ of errors are located within $3\sigma$ around the
actual range between a ray and an obstacle, assuming errors are distributed
normally. The minimum standard deviation is reported for VELODYNE sensors
\cite{velodyne_datasheet}; the rest are reported for price-appealing but
disturbance-laden RPLIDAR \cite{a2m8_datasheet} and YDLIDAR
\cite{ydlidar_datasheets} sensors. The size of the input scans was set to
$N_s=360$ rays. The minimum and maximum map oversampling rates of FSM were set
to $(2^{\nu_{\min}},2^{\nu_{\max}}) = (2^0,2^3)$.  The number of iterations of
the translational component at each map sampling degree $\nu$ was set at $I =
5\nu$. The orientation convergence threshold was set to $\varepsilon_{\delta p}
= 10$e-$5$. Maximal displacements $\overline{\delta}_{xy}$ and
$\overline{\delta}_\theta$ were chosen as such by prior art tests
\cite{Censi2008b}.  For each experiment all algorithms ran for $E = 10$ times
across all instances of $D$; therefore each method underwent a total of $10
\times 45402 \times 6 \times 5 \sim O(10^8)$ experiments. The orientation and
position error distributions reported below are those across all $E \cdot |D|$
experiments of the same configuration.  All experiments and algorithms were run
in C++, on a single thread, on a machine with a CPU frequency of $4.0$ GHz. The
implementations of PLICP, NDT, FastGICP, FastVGICP, and NDT-PSO were taken from
\cite{implementations}.
