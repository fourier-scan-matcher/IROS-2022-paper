\begin{figure*}
  \vspace{0.2cm}
  \begin{framed}
  \vspace{-0.75cm}\hspace{-0.75cm}
    \subfloat{\input{./figures/experiments/orientation_errors.tex}}
    \qquad \hspace{-1.25cm}
    \subfloat{\input{./figures/experiments/position_errors.tex}}
    \vspace{-2.5cm}
    \caption{\small Distribution of orientation and position errors across a
             range of maximal positional and orientational displacements, for
             progressively larger sensor measurement noise levels sd. Each
             boxplot represents $10$ iterations over $\sum|D_k| \approx
             45$$\cdot$$10^3$ random scan pairs for each configuration,
             where $k=1,\dots,5$ is the dataset index. Dots signify mean
             errors. FSM's errors are largely independent of the initial
             displacement of scans for a given level of sensor noise}%
    \label{fig:error_distributions}%
\end{framed}
\end{figure*}

Figure \ref{fig:error_distributions} shows the distribution of rotation and
translation errors across all experiments for all tested algorithms.  FSM's
position and orientation errors are equal to or lower than the most accurate
method for each displacement and sensor noise configuration. As displacements
and sensor noise levels increase, its errors increase at a lower rate than
those of any tested method. The magnitude of FSM's errors is largely
independent of the displacement of the two input scans for a given level of
sensor noise (fig.  \ref{fig:laser_odometry}). In terms of orientation,
$72\%$-$74\%$ of FSM's errors resulted below $\gamma / 2^{\nu_{\max}+1} =
0.0625$ deg when $\sigma_R = 0.01$ m, and $33\%$-$36\%$ when $\sigma_R = 0.20$
m.  The juxtaposition of the six methods' errors at high levels of sensor noise
highlight the robustness afforded to FSM by the Discrete Fourier transform and
its properties.  In terms of execution time, PLICP ranged between $4.8$-$17.5$
ms, NDT $8.1$-$19.9$ ms, FastGICP $3$-$9$ ms, FastVGICP $3.8$-$6.8$ ms, NDT-PSO
$190$-$200$ ms, and FSM between $17.7$ and $23.7$ ms.  The measurement
frequency of modern LIDAR sensors ranges from $12$-$20$ Hz; therefore FSM runs
in real time in modern processors.
