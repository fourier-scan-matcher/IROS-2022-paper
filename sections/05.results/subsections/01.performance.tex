Figure \ref{fig:error_distributions} shows the distribution of rotation and
translation errors across all experiments for all tested algorithms.

\begin{figure*}\hspace{-0.25cm}
    \subfloat{\input{./figures/experiments/orientation_errors.tex}}
    \qquad \hspace{-1.5cm}
    \subfloat{\input{./figures/experiments/position_errors.tex}}
    \vspace{-2.5cm}
    \caption{\small Distribution of orientation and position errors across a
             range of maximal positional and orientational displacements, for
             progressively larger sensor measurement noise levels. FSM's errors
             are largely independent of the initial displacement of scans for a
             given level of sensor noise}%
    \label{fig:error_distributions}%
\end{figure*}

At small location and orientation displacements between the two input scans
($\overline{\delta}_{xy} \leq $ 0.05 m,
$\overline{\delta}_\theta \leq 2^\circ$), CSM outperforms NDT and FSM for low
levels of sensor noise ($\sigma_R \leq 0.01$ m). However, as noise increases,
FSM starts exhibiting greater robustness and accuracy than CSM. At greater
location and orientation displacements
($\overline{\delta}_{xy} > 0.05$ m, $\overline{\delta}_\theta > 2^\circ$), FSM
is able to maintain errors equal to or lower than CSM across the entirety of
the spectrum of tested noise levels. Compared to NDT, FSM exhibits greater
accuracy across all tested configurations. The magnitude and variability of
FSM's errors for a given level of sensor noise is independent of the
displacement of the two input scans (fig. \ref{fig:laser_odometry}). The
juxtaposition of the three methods' pose errors at high levels of sensor noise
highlight the robustness afforded to FSM by the Discrete Fourier transform and
its properties. With regard to FSM's orientation errors, $71.0\%$-$79.6\%$ of
all final orientation errors resulted under $\gamma / 2^{\nu_{\max}+1} =
0.0011$ rad when $\sigma_R = 0.0$ m. In terms of execution time, CSM ranged
from $4.8$ to $20.5$ ms, NDT from $8.1$ to $19.9$ ms, and FSM from $13.2$ to
$16.7$ ms. Therefore FSM's exhibits the least variability to sensor noise and
locational and orientational displacement in terms of runtime.  The measurement
frequency of modern LIDAR sensors ranges from $12$-$20$ Hz; therefore FSM runs
in real time in modern processors.
