Scan-matching with the use of a 2D LIDAR sensor began with Iterative Dual
Correspondences (IDC) \cite{FengLu1994}, an algorithm incorporating elements of
the Iterative Closest Point (ICP) algorithm \cite{Besl1992b}. The latter and
its variants, e.g.  \cite{Pfister,Chetverikov,Censi2008b,Segal2009}, have
become the de facto scan-matching algorithms in 2D and 3D settings, with
research using ICP being still ongoing
\cite{Wang2018a,Tian2019a,Marchel2020,Koide2021}. In particular, PLICP
\cite{Censi2008b} has been widely adopted due to its increased accuracy among
ICP variants, and the availability of its source code. ICP and its variants,
however, exhibit varying performance \cite{Donoso2017a}, limited by the noise
level in the input scans, the choice of prior, and the configuration of the
parameters governing their response (a detailed account may be found in
\cite{Filotheou2022a}). The vast majority of all matching methods adopt ICP's
approach of establishing correspondences between the two input scans, using
various assumptions, mechanisms (e.g. the Normal Distributions Transform
\cite{Biber,Bouraine2020} which models points to distributions), and types of
sources (e.g. features instead of points \cite{Wang2018b}; a detailed review of
scan-matching methods may be found in \cite{Filotheou2020a}). The major problem
with establishing correspondences is that the process becomes more inefficient
and error-prone as measurement noise or displacement between sensor poses
increases. By contrast, the method introduced in this paper does not operate by
establishing correspondences, and its accuracy does not depend on the
rotational or translational displacement between sensor poses for a given level
of measurement noise.

FSM's rotational component is most akin to those of
\cite{Yu2018} and \cite{Jiang2018}. They use Phase-Only Matched Filtering
(POMF) \cite{Qin-ShengChen1994} in both rotation and translation components;
the former in one dimension and the latter in two dimensions. In the latter,
the requirements for a real-time solution and adequate accuracy cannot be
fulfilled simultaneously due to the inability to balance high grid resolution
(and therefore high accuracy) with regular sensor updates. The former
alleviates this limitation by operating in one dimension, but suffers from the
same causes, namely, discretisation errors. Whereas the latter is dependent on
the grid's resolution, the former is dependent on the sensor's immutable angle
increment.  In both methods, both the rotational and translational components
are affected, but no mitigation technique is employed to decrease the errors of
either their components. By contrast, the method introduced in this paper
addresses all the above issues by (a) aiming to extricate the orientation error
from the sensor's angle increment, (b) employing a continuous-space translation
method, and (c) fulfilling the real-timeness constraint.
