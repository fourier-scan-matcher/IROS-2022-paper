%%%%%%%%%%%%%%%%%%%%%%%%%%%%%%%%%%%%%%%%%%%%%%%%%%%%%%%%%%%%%%%%%%%%%%%%%%%%%%%%
\begin{definition}
  \label{def:definition_1}
  \textit{Definition of a range scan captured from a conventional 2D LIDAR
  sensor.} A conventional 2D LIDAR sensor provides a finite number of ranges,
  i.e. distances to objects within its range, on a horizontal cross-section of
  its environment, at regular angular and temporal intervals, over a defined
  angular range \cite{lidar}. We define a range scan $\mathcal{S}$, consisting
  of $N_s$ rays over an angular range $\lambda$, to be an ordered map
  $\mathcal{S} : \Theta \rightarrow \mathbb{R}_{\geq 0}$, where $\Theta =
  \{\theta_n \in [-\frac{\lambda}{2}, +\frac{\lambda}{2}) : \theta_n =
  -\frac{\lambda}{2} + \lambda \frac{n}{N_s}$, $n = 0,1,\dots, N_s$$-$$1$$\}$.
  Angles $\theta_n$ are expressed relative to the sensor's heading, in the
  sensor's frame of reference. A LIDAR sensor's angle increment $\gamma$ is the
  angular distance between two consecutive rays: $\gamma \triangleq
  \dfrac{\lambda}{N_s}$.
\end{definition}

%%%%%%%%%%%%%%%%%%%%%%%%%%%%%%%%%%%%%%%%%%%%%%%%%%%%%%%%%%%%%%%%%%%%%%%%%%%%%%%%
%\begin{definition}
  %\label{def:definition_2}
  %\textit{Scan-matching (or scan-to-scan matching) using a 2D LIDAR sensor} (adapted for use in
  %two dimensions from \cite{plicp}).
  %Let two range scans as defined by Definition \ref{def:definition_1},
  %$\mathcal{S}_R$ and $\mathcal{S}_V$, be captured from a LIDAR
  %sensor operating in the same environment at both capturing times. Let
  %$\bm{p}_V(x_V,y_V,\theta_V)$ be the pose from
  %which the sensor captured $\mathcal{S}_V$, expressed in some coordinate
  %system (usually a past pose estimate of the sensor). The objective of
  %scan-matching in two dimensions is to find the roto-translation
  %$\bm{q} = (\bm{t}, \theta)$, $\bm{t} = (\Delta x, \Delta y)$ that minimises
  %the distance of the endpoints of $\mathcal{S}_V$ roto-translated by
  %$\bm{q}$ to their projection on $\mathcal{S}_R$. Denoting the
  %endpoints of $\mathcal{S}_V$ by $\{\bm{p}_V^i\}$, in formula:
  %\begin{align}
    %\underset{\bm{q}}{\min} \sum\limits_i \Big\| \bm{p}_V^i \oplus \bm{q} - \prod \{ \mathcal{S}_R, \bm{p}_V^i \oplus \bm{q} \}\Big\|^2
    %\label{eq:s2sm_def}
  %\end{align}
  %The symbol ``$\oplus$" denotes the roto-translation operator
  %$\bm{p}_V^i \oplus (\bm{t}, \theta) \triangleq \bm{R}(\theta) \bm{p}^i_V + \bm{t}$,
  %where $\bm{R}(\theta)$ is the 2D rotation matrix for argument angle $\theta$,
  %and $\prod\{\mathcal{S}_R, \bm{p}_V^i \oplus \bm{q} \}$ denotes
  %the Euclidean projector on $\mathcal{S}_R$.
%\end{definition}


%%%%%%%%%%%%%%%%%%%%%%%%%%%%%%%%%%%%%%%%%%%%%%%%%%%%%%%%%%%%%%%%%%%%%%%%%%%%%%%%
\begin{definition}
  \label{def:definition_3}
  \textit{Definition of a map-scan.}
  A map-scan is a virtual scan that encapsulates the same pieces of information
  as a scan derived from a physical sensor. Only their underlying operating
  principle is different due to the fact the map-scan refers to distances to
  obstacles within a point-set, usually referred to as the map, rather than
  within an real environment. A map-scan is derived by means of locating
  intersections of rays emanating from the estimate of the sensor's pose and
  the boundaries of the map.
\end{definition}
