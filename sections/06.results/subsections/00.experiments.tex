The experimental procedure was conducted using a benchmark dataset $D$
consisting of $|D| = 778$ laser scans obtained from a Sick range-scan sensor
mounted on a robotic wheel-chair \cite{dataset_link}. For
each scan $D^d$, $d = 1,\dots,778$, the dataset reports one range scan of $360$
range measurements and the pose from which it was captured
$\bm{r}^d(x,y,\theta)$.  The same dataset was used to evaluate the performance
of IDC \cite{LuMilios}, ICP, and MBICP in \cite{mbicp}, and that of PLICP and
the joint method PLICP$\circ$GPM during scan-matching experiments. In
\cite{plicp} the latter was found to be the best-performing among the five
correspondence-finding scan-matching methods. Therefore, for purposes of
comparison against correspondence-finding scan-matching methods, the
experimental procedure is extended to PLICP$\circ$GPM. This method shall be
denoted hereafter by the acronym CSM. In the same vein, for purposes of
comparison against correspondenceless scan-matching methods, the same
experimental procedure is extended to the NDT scan-matching method \cite{ndt1}.

The experimental setup is the following. The rays of each dataset instance
$D^d$ are first projected to the $x-y$ plane around $\bm{r}^d$. The dataset's
scans are not panoramic, therefore the remaining space is filled with a
semicircular arc that joins the scan's two extreme ends. Similar fashions for
closing-off the environment have been found equivalent with respect to the
performance of the tested methods. The resulting point-set is regarded as the
environment $\bm{W}^d$ in which the range sensor operates.  Then the pose
$\bm{p}_0^d$ from which $\mathcal{S}_0^d$ is captured is generated randomly
within the polygon formed by $\bm{W}^d$. The pose $\bm{p}_1^d$ from which the
sensor captured $\mathcal{S}_1$ is then obtained by perturbing the components
of $\bm{p}_0^d$ with quantities extracted from uniformly distributed error
distributions $U_{xy}(-\overline{\delta}_{xy}, \overline{\delta}_{xy})$,
$U_{\theta}(-\overline{\delta}_{\theta}, \overline{\delta}_{\theta})$;
$\overline{\delta}_{xy}$, $\overline{\delta}_\theta$ $\in \mathbb{R}_{\geq 0}$.

Range scans $\mathcal{S}_0^d$ and $\mathcal{S}_1^d$ are then computed by
locating the intersection points between $N_s$ rays emanating from $\bm{p}_0^d$
and $\bm{p}_1^d$, respectively, and the polygon formed by $\bm{W}^d$ across an
angular field of view $\lambda = 2\pi$. The inputs to CSM, NDT, and FSM are
then set to $\mathcal{S}_0^d$ and $\mathcal{S}_1^d$. Their output is
$\bm{p}_1^{\prime d}$. The roto-translation
$\hat{\bm{T}}^d = \bm{p}_1^{\prime d}$ is the estimate of the motion
$\bm{T}^d = \bm{p}_1^d - \bm{p}_0^d$ of the range sensor. The criterion on
which the evaluation of all experiments rests is the $2$-norm of the total pose
displacement error
\begin{align}
  e^d &= \| \bm{T}^d - \hat{\bm{T}}^d \|_2
  \label{eq:rototranslation_error}
\end{align}

For every pose estimate $\bm{p}_1^{\prime d}$ outputted by
each algorithm, $d = 1,2,\dots,|D|$, its offset from the actual pose
$\bm{p}_1^d$ is recorded in the form of the $2$-norm total error. The pose
errors of one experiment are then averaged. The pose error distributions
reported below are those of mean errors across $E$ experiments of the same
configuration.

In order to test for the performance of the proposed method with use of real
sensors, five levels of noise acting on the range measurements of the scans are
tested. The range measurements are perturbed by zero-mean normally-distributed
noise with standard deviation $\sigma_R \in \{0.0, 0.01, 0.03, 0.05, 0.10\}$ m.
The non-zero values of tested standard deviations were calculated from
commercially available panoramic LIDAR scanners by identifying the magnitude of
their reported maximum range errors and dividing it by a factor of three. The
rationale is that $99.73\%$ of errors are located within $3\sigma$ around the
actual range between a ray and an obstacle, assuming errors are distributed
normally. The minimum standard deviation $\sigma_R = 0.01$ m is reported for
VELODYNE sensors \cite{velodyne_datasheet}; the rest are reported for
price-appealing but disturbance-laden sensors, e.g. the RPLIDAR A2M8, or the
YDLIDAR G4, TG30, and X4 scanners \cite{a2m8_datasheet}-\cite{x4_datasheet}. The
size of the input scans was set to $N_s=360$ rays. The minimum and maximum map
oversampling rates of FSM were set to $(\mu_{\min},\mu_{\max}) =
(2^{\nu_{\min}},2^{\nu_{\max}}) = (2^0,2^3)$. The number of iterations of the
translational component at each map sampling degree $\nu$ was set at $I =
2\nu$. The orientation convergence threshold was set to $\varepsilon_{\delta p}
= 10$e-$5$. Maximal displacements $\overline{\delta}_{xy}$ and
$\overline{\delta}_\theta$ were chosen as such by prior art tests \cite{plicp}.
For each experiment FSM, CSM, and NDT ran for $E = 100$ times across all
instances of $D$. Therefore each method underwent a total of
$100 \times 778 \times 6 \times 5 \sim O(10^6)$ experiments.
All experiments and algorithms were run serially, in C++, on
a single thread, on a machine with a CPU frequency of $4.0$ GHz. The
implementations of CSM and NDT were taken from \cite{csm_implementation} and
\cite{ndt_implementation} respectively.
