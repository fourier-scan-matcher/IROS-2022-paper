Let the assumptions of Problem \ref{prob:the_problem} hold. Assume now
that $\mathcal{S}_0$ and $\mathcal{S}_1$ were captured from different
positions in the same environment but with the same orientation
relative to a fixed reference frame. Let $\mathcal{S}_0$ be projected onto
the $x-y$ plane around pose $\bm{s}(0,0,0)$,
producing point-set $\bm{M}_L$. Assuming that $\mathcal{S}_1$ was captured in a
neighbourhood of $\mathcal{S}_0$, then $\bm{M}_L$ is a perturbed local map of
the environment with respect to sensor measurement $\mathcal{S}_1$. Aside from
measurement noise, this perturbation manifests due to the finiteness of the
sensor's angle increment and to the fact that different portions of the
environment are perceptible and therefore measurable from different locations
\cite{Olson2009}. The nature of these perturbations on map-scans captured within
$\bm{M}_L$ is additive and finite. Under these assumptions the problem of
(scan-)matching scan $\mathcal{S}_1$ to scan $\mathcal{S}_0$ may be transformed
into a problem of scan--to--map-scan matching, where the aim is registering
scan $\mathcal{S}_1$ to map $\bm{M}_L$: i.e. estimating the pose $\bm{p}_1$
from where $\mathcal{S}_1$ was captured within $\bm{M}_L$. Theorem
\ref{prop:theorem_without_disturbance} \cite{Filotheou2022} guarantees that the
error of the location estimate between the poses from which the two scans were
captured is bounded in a neighbourhood of the origin, when $\hat{\bm{p}}_1 =
\bm{s}$.


\begin{theorem}
  \label{prop:theorem_without_disturbance}
  Let a panoramic 2D range scan $\mathcal{S}_1$ be captured from a physical
  range sensor from unknown pose $\bm{p}_1 = (\bm{l}_1,\theta_1)$, $\bm{l}_1 = (x_1,y_1)$.
  Let $\bm{M}_L$ be the map, i.e. the projection of panoramic 2D range scan
  $\mathcal{S}_0$ from $\bm{s}(0,0,0)$. Let a
  pose estimate $\hat{\bm{p}}_1 = (\hat{\bm{l}}_1, \hat{\theta}_1)$ reside in the
  neighbourhood of $\bm{p}_1$ in the map's frame of reference. Additionally, let
  $\hat{\theta}_1 = \theta_1$. Assume that $\mathcal{S}_0$ and/or $\mathcal{S}_1$
  are affected by additive, bounded disturbances. Then,
  treating the estimate of the location of the sensor as a state variable
  $\hat{\bm{l}}_1[k] = [\hat{x}_1[k], \hat{y}_1[k]]^\top$ and updating it according
  to the difference equation $\hat{\bm{l}}_1[k+1] = \hat{\bm{l}}_1[k] + \bm{u}[k]$,
  where $\hat{\bm{l}}_1[0] = \hat{\bm{l}}_1 = [\hat{x}_1, \hat{y}_1]^{\top}$,
  i.e. the supplied initial location estimate, with $\bm{u}$ being
  \begin{align}
    \bm{u}[k] = \dfrac{1}{N_s}
    \begin{bmatrix}
      \cos\hat{\theta}_1 & \sin\hat{\theta}_1 \\
      \sin\hat{\theta}_1 & - \cos\hat{\theta}_1
    \end{bmatrix}
    \begin{bmatrix}
      X_{1,r}\big(\mathcal{S}_1, \mathcal{S}_0|_{\bm{\hat{p}}_1[k]}\big) \vspace{0.2cm} \\
      X_{1,i}\big(\mathcal{S}_1, \mathcal{S}_0|_{\bm{\hat{p}}_1[k]}\big)
    \end{bmatrix}
    \label{eq:control_vector_without_disturbance}
  \end{align}
  where $X_{1,r}(\cdot)$ and $X_{1,i}(\cdot)$ are, respectively, the real and
  imaginary parts of the complex quantity $X_1$:
  \begin{align}
    X_1\big(\mathcal{S}_1, \mathcal{S}_0|_{\bm{\hat{p}}_1[k]}\big)
      %= &X_{1,r}\big(\mathcal{S}_R, \mathcal{S}_V|_{\bm{\hat{p}}[k]}\big)
      %+ i X_{1,i}\big(\mathcal{S}_R, \mathcal{S}_V|_{\bm{\hat{p}}[k]}\big) \nonumber \\
      = &\sum\limits_{n=0}^{N_s-1}(\mathcal{S}_1[n] - \mathcal{S}_0[n]|_{\bm{\hat{p}}_1[k]}) \cdot e^{-i \frac{2 \pi n}{N_s}} \label{eq:X1}
  \end{align}
  where $\mathcal{S}_1[n]$ and $\mathcal{S}_0[n]|_{\bm{\hat{p}}_1[k]}$ are,
  respectively, the ranges of the $n$-th ray of scan $\mathcal{S}_1$, and
  map-scan $\mathcal{S}_0|_{\bm{\hat{p}}_1[k]}$ captured via raycasting the map
  $\bm{M}_L$ from $\bm{\hat{p}}_1[k] = (\hat{\bm{l}}_1[k],
  \hat{\theta}_1)$---then $\hat{\bm{l}}_1[k]$ is uniformly bounded for $k \geq
  k_0$ and uniformly ultimately bounded in a neighbourhood of $\bm{l}_1$. Its
  size depends on the suprema of the disturbance corrupting the range
  measurements of the two scans.
\end{theorem}


\begin{remark}
  \label{remark:loc_prop_or}
  Without loss of generality, subsequent to the application of Theorem
  \ref{prop:theorem_without_disturbance}, the location error is proportional to
  the orientation error.
\end{remark}

Let $\hat{\bm{p}}_1^\prime$ denote the resulting pose estimate of $\bm{p}_1$
in $\bm{M}_L$. Then
$\hat{\bm{T}} = \hat{\bm{p}}_1^\prime - \bm{s} = \hat{\bm{p}}_1^\prime$ is the
estimate of the 3D rigid transformation of the sensor as it moved from the pose
where it captured $\mathcal{S}_0$ to that where it captured $\mathcal{S}_1$.
