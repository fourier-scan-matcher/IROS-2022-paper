Let the assumptions of Problem \ref{prob:the_problem} be standing. Assume that
the two scans were captured from the same location but from different
orientations. Denoting with $\mathcal{F}\{\mathcal{S}\}$ the Discrete Fourier
Transform (DFT) of signal $\mathcal{S}$, with $\mathcal{F}^{-1}\{\mathcal{S}\}$
its inverse, with $\bm{c}^\ast$ the conjugate of complex $\bm{c}$, and with
$|\bm{c}|$ its magnitude, calculate
$Q_{\mathcal{S}_0, \mathcal{S}_1}$:
\begin{align}
  Q_{\mathcal{S}_0, \mathcal{S}_1} \triangleq \dfrac{\mathcal{F}\{\mathcal{S}_0\}^{\ast} \cdot \mathcal{F}\{\mathcal{S}_1\}}{|\mathcal{F}\{\mathcal{S}_0\}| \cdot |\mathcal{F}\{\mathcal{S}_1\}|}
  \label{eq:Q}
\end{align}
on the basis that if space is sampled sufficiently densely, for
$k,\xi \in \mathbb{Z}$: $k,\xi \in [0, N_s-1]$:
\begin{align}
  \mathcal{S}_0[k] &\simeq \mathcal{S}_1[(k - \xi) \mod N_s] \Leftrightarrow \nonumber \\
  \mathcal{F}\{\mathcal{S}_0\}(u) &\simeq e^{-j 2\pi \xi u / N_s} \cdot \mathcal{F}\{\mathcal{S}_1\}(u) \nonumber
\end{align}
and, therefore, since $2\pi \dfrac{\xi}{N_s} = \xi \gamma$: $Q_{\mathcal{S}_0, \mathcal{S}_1}(u)  \simeq e^{-j \xi \gamma u}$.

The inverse of $Q_{\mathcal{S}_0, \mathcal{S}_1}$ is a Kronecker
$\delta$-function
$q_{\mathcal{S}_0, \mathcal{S}_1} = \mathcal{F}^{-1}\{Q_{\mathcal{S}_0, \mathcal{S}_1}\}$
centered at $\xi = \operatorname*{arg\,max}\limits_u \ q_{\mathcal{S}_0, \mathcal{S}_1}(u)$.
If the difference in orientation between the two scans is $\Delta\theta$, then
$\Delta\theta = \xi\gamma + \delta\theta$, where
$\mod(|\delta\theta|, \gamma) = \lambda \in [0,\frac{\gamma}{2}]$. Therefore for
a given number of emitted rays $N_s$ there remains an unresolved orientation
error $|\delta\theta| \leq \gamma/2$. The contribution of this error to the
scan-matching error is two-fold, as its existence is also propagated to the
location estimation method. A method for further reduction of
the orientation error is presented in the following.

Let $\mathcal{S}_0$ be projected onto the $x-y$ plane around an arbitrary but
fixed pose $\bm{s}(x_s, y_s, \theta_s)$, producing point-set $\bm{M}_R$.
$\bm{M}_R$ will hereafter be referred to as the map. Then compute $2^\nu$
map-scans (def. \ref{def:definition_3}) $\mathcal{S}_0^k$,
$k = 0,\dots,2^\nu-1$, starting from orientation $\theta_s$, at $\gamma / 2^\nu$
angular increments. Then the orientation estimation process is carried out once
between $\mathcal{S}_1$ and map scan $\mathcal{S}_0^k$ taken from orientation
$\theta_0^k = \theta_s + k \cdot \gamma / 2^\nu$, for a total of $2^\nu$ times.
An alignment metric between the $k$-th map scan $\mathcal{S}_0^k$ and scan
$\mathcal{S}_1$ is computed according to
\begin{align}
  \text{PD}_k = \dfrac{2 \max q_{\mathcal{S}_0^k,\mathcal{S}_1}}{\max q_{\mathcal{S}_0^k,\mathcal{S}_0^k} + \max q_{\mathcal{S}_1,\mathcal{S}_1}}
  \label{eq:pd}
\end{align}

The Percent Discrimination metric PD$_k$ $\in [0,1]$, and is proportional to
the degree of alignment between map-scan $\mathcal{S}_0^k$ and
scan $\mathcal{S}_1$, across all $2^\nu$ map-scans $\mathcal{S}_0^k$.
The above analysis is the equivalent of the 2D Fourier-Mellin Invariant
matching in one dimension \cite{Qin-ShengChen1994}.

Let now $K$ denote the index of the $k$-th map scan
$\mathcal{S}_0^K$ scoring the highest PD$_k$: $\text{PD}_K =
\max \{\text{PD}_k\}$, $k = 0,\dots,2^\nu-1$. Let also $\Xi$ denote the integer
multiple of angle increments $\gamma$ by which $\mathcal{S}_0^K$
should be rotated counter-clockwise in order to achieve PD$_K$:
$\Xi = \arg\max\ q_{\mathcal{S}_0^K, \mathcal{S}_1}$.  Then the sensor's
orientation difference becomes
$\Delta\theta = \Xi\gamma + K \cdot \gamma/2^\nu + \delta\theta^\prime$.

If map-scans $\mathcal{S}_0^k$ were computed by raycasting the map of the
environment instead of $\bm{M}_R$ then the residual and unresolved orientation
error $|\delta\theta^\prime| \in [0,\gamma / 2^{1+\nu}]$. In this case, however,
$\bm{M}_R$ is an approximation of the environment's map in the locality of
$\bm{p}_0$. Depending on the magnitude of the sensor's angle increment and
the arbitrariness of the environment, this approximation may be viewed as
induced local perturbations in the map of the environment. This holds true
in the general case as well, where $\mathcal{S}_0$ and $\mathcal{S}_1$ are
captured from different locations. Therefore attaining
$|\delta\theta^\prime| \leq \gamma / 2^{1+\nu}$ may not always be possible
for all combinations of environments and sensor angle increments.
