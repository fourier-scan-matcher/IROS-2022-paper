Consider a robot capable of motion, equipped with a Light Detection and Ranging
sensor (LIDAR), capturing a measurement $\mathcal{S}_0$ at time $t_0$ from pose
$\bm{p}_0$ in some reference frame. The robot then moves to pose $\bm{p}_1$ at
time $t_1$ at which time it captures measurement $\mathcal{S}_1$. Provided
overlap between the two scans, estimating the rigid-body transformation
$\bm{T}$ that projects the endpoints of $\mathcal{S}_1$ to those of
$\mathcal{S}_0$ with the least error is known as scan-matching. The solution to
the scan-matching problem is central to methods of Localisation
\cite{lidar_localisation_1}\cite{lidar_localisation_2}, Navigation
\cite{lidar_navigation_1}-\cite{lidar_navigation_4}, and Simultaneous
Localisation and Mapping (SLAM) \cite{lidar_slam_1}-\cite{lidar_slam_5}, as
$\bm{T}$ is the rigid-body transformation $\bm{p}_1$$-$$\bm{p}_0$: i.e. the
solution to scan-matching provides localisation information at time $t_1$,
relative to $\bm{p}_0$. For this reason, along with the high measurement
accuracy of LIDAR measurements, scan-matching is also used as a means to
improving, providing, or substituting odometric measurements (where available),
as the latter are prone to unbounded and unpredictable tire and wheel
slippage \cite{olson}-\cite{lidar_odom_3}.

LIDAR sensors with a field of view of $360^\circ$, i.e. panoramic sensors, were
for years constrained to high price ranges, and most provided 3D measurements.
Therefore research on scan-matching with 2D LIDAR sensors mostly focused on
non-panoramic sensors, with scan matching methods being used without distinction
with regard to field of view. In recent years, however, price-appealing
2D LIDAR sensors have emerged, but at the cost of increased measurement
uncertainty. The introduction of these sensors warrants targeted research
into scan-matching with the use of panoramic LIDAR sensors, due to (a) the
afforded periodicity of the range signal, and (b) the need of
addressing the high levels of measurement noise with regard to the
transformation errors of current scan-matching algorithms.

This paper introduces a real-time method specifically targeting the matching of
2D panoramic range scans, whose error is invariant to angular and locational
displacement for a given level of measurement noise.  The central contributions
of the paper are:

\begin{itemize}
  \item To the best of the author's knowledge, the first method explicitly
        addressing the matching of panoramic 2D range scans
  \item The extrication from the need of a prior transformation estimate
  \item The introduction of a method that aims at reducing the orientation
        error to lower than the sensor's angle increment compared to relevant
        prior work
  \item The parameter set needed by the proposed method is intuitive, smaller
        than those of established methods, and trades accuracy for execution
        time
  \item The thorough evaluation of the proposed method against two established
        scan-matching algorithms in common use, using measurement noise levels
        from common, commercially available sensors
\end{itemize}

\begin{figure}[]\centering
  \vspace{-1.7cm}
  \definecolor{aa}{RGB}{227,26,28}
\definecolor{ab}{RGB}{56,125,184}
\definecolor{ac}{RGB}{77,176,74}
\definecolor{ak}{RGB}{0,0,0}

% GNUPLOT: LaTeX picture with Postscript
\begingroup
  \makeatletter
  \providecommand\color[2][]{%
    \GenericError{(gnuplot) \space\space\space\@spaces}{%
      Package color not loaded in conjunction with
      terminal option `colourtext'%
    }{See the gnuplot documentation for explanation.%
    }{Either use 'blacktext' in gnuplot or load the package
      color.sty in LaTeX.}%
    \renewcommand\color[2][]{}%
  }%
  \providecommand\includegraphics[2][]{%
    \GenericError{(gnuplot) \space\space\space\@spaces}{%
      Package graphicx or graphics not loaded%
    }{See the gnuplot documentation for explanation.%
    }{The gnuplot epslatex terminal needs graphicx.sty or graphics.sty.}%
    \renewcommand\includegraphics[2][]{}%
  }%
  \providecommand\rotatebox[2]{#2}%
  \@ifundefined{ifGPcolor}{%
    \newif\ifGPcolor
    \GPcolorfalse
  }{}%
  \@ifundefined{ifGPblacktext}{%
    \newif\ifGPblacktext
    \GPblacktexttrue
  }{}%
  % define a \g@addto@macro without @ in the name:
  \let\gplgaddtomacro\g@addto@macro
  % define empty templates for all commands taking text:
  \gdef\gplbacktext{}%
  \gdef\gplfronttext{}%
  \makeatother
  \ifGPblacktext
    % no textcolor at all
    \def\colorrgb#1{}%
    \def\colorgray#1{}%
  \else
    % gray or color?
    \ifGPcolor
      \def\colorrgb#1{\color[rgb]{#1}}%
      \def\colorgray#1{\color[gray]{#1}}%
      \expandafter\def\csname LTw\endcsname{\color{white}}%
      \expandafter\def\csname LTb\endcsname{\color{black}}%
      \expandafter\def\csname LTa\endcsname{\color{black}}%
      \expandafter\def\csname LT0\endcsname{\color[rgb]{1,0,0}}%
      \expandafter\def\csname LT1\endcsname{\color[rgb]{0,1,0}}%
      \expandafter\def\csname LT2\endcsname{\color[rgb]{0,0,1}}%
      \expandafter\def\csname LT3\endcsname{\color[rgb]{1,0,1}}%
      \expandafter\def\csname LT4\endcsname{\color[rgb]{0,1,1}}%
      \expandafter\def\csname LT5\endcsname{\color[rgb]{1,1,0}}%
      \expandafter\def\csname LT6\endcsname{\color[rgb]{0,0,0}}%
      \expandafter\def\csname LT7\endcsname{\color[rgb]{1,0.3,0}}%
      \expandafter\def\csname LT8\endcsname{\color[rgb]{0.5,0.5,0.5}}%
    \else
      % gray
      \def\colorrgb#1{\color{black}}%
      \def\colorgray#1{\color[gray]{#1}}%
      \expandafter\def\csname LTw\endcsname{\color{white}}%
      \expandafter\def\csname LTb\endcsname{\color{black}}%
      \expandafter\def\csname LTa\endcsname{\color{black}}%
      \expandafter\def\csname LT0\endcsname{\color{black}}%
      \expandafter\def\csname LT1\endcsname{\color{black}}%
      \expandafter\def\csname LT2\endcsname{\color{black}}%
      \expandafter\def\csname LT3\endcsname{\color{black}}%
      \expandafter\def\csname LT4\endcsname{\color{black}}%
      \expandafter\def\csname LT5\endcsname{\color{black}}%
      \expandafter\def\csname LT6\endcsname{\color{black}}%
      \expandafter\def\csname LT7\endcsname{\color{black}}%
      \expandafter\def\csname LT8\endcsname{\color{black}}%
    \fi
  \fi
    \setlength{\unitlength}{0.0500bp}%
    \ifx\gptboxheight\undefined%
      \newlength{\gptboxheight}%
      \newlength{\gptboxwidth}%
      \newsavebox{\gptboxtext}%
    \fi%
    \setlength{\fboxrule}{0.5pt}%
    \setlength{\fboxsep}{1pt}%
\begin{picture}(5000.00,5000.00)%
    \gplgaddtomacro\gplbacktext{%
      \colorrgb{0.15,0.15,0.15}%
      \put(391,1564){\makebox(0,0)[r]{\strut{}$6.0$}}%
      \colorrgb{0.15,0.15,0.15}%
      \put(391,1988){\makebox(0,0)[r]{\strut{}$8.0$}}%
      \colorrgb{0.15,0.15,0.15}%
      \put(391,2413){\makebox(0,0)[r]{\strut{}$10.0$}}%
      \colorrgb{0.15,0.15,0.15}%
      \put(391,2837){\makebox(0,0)[r]{\strut{}$12.0$}}%
      \colorrgb{0.15,0.15,0.15}%
      \put(391,3262){\makebox(0,0)[r]{\strut{}$14.0$}}%
      \colorrgb{0.15,0.15,0.15}%
      \put(391,3687){\makebox(0,0)[r]{\strut{}$16.0$}}%
      \colorrgb{0.00,0.00,0.00}%
      \put(672,1280){\makebox(0,0){\strut{}$-2.0$}}%
      \colorrgb{0.00,0.00,0.00}%
      \put(1096,1280){\makebox(0,0){\strut{}$0.0$}}%
      \colorrgb{0.00,0.00,0.00}%
      \put(1521,1280){\makebox(0,0){\strut{}$2.0$}}%
      \colorrgb{0.00,0.00,0.00}%
      \put(1945,1280){\makebox(0,0){\strut{}$4.0$}}%
      \colorrgb{0.00,0.00,0.00}%
      \put(2370,1280){\makebox(0,0){\strut{}$6.0$}}%

      \put(1120,4104){\makebox(0,0){\strut{}{\color{ak}{\rule[0.6mm]{0.5cm}{0.5mm}}} \footnotesize sensor trajectory}}
      \put(2470,4104){\makebox(0,0){\strut{}{\color{aa}{\rule[0.6mm]{0.5cm}{0.5mm}}} \footnotesize PLICP}}
      \put(3370,4104){\makebox(0,0){\strut{}{\color{ab}{\rule[0.6mm]{0.5cm}{0.5mm}}} \footnotesize NDT}}
      \put(4240,4104){\makebox(0,0){\strut{}{\color{ac}{\rule[0.6mm]{0.5cm}{0.5mm}}} \footnotesize FSM}}
      \put(1500,1004){\makebox(0,0){\strut{} \footnotesize Frequent measurements}}
      \put(3600,1004){\makebox(0,0){\strut{} \footnotesize Infrequent measurements}}
    }%
    \gplgaddtomacro\gplfronttext{%
    }%
    \gplgaddtomacro\gplbacktext{%
      \colorrgb{0.00,0.00,0.00}%
      \put(2822,1280){\makebox(0,0){\strut{}$-2.0$}}%
      \colorrgb{0.00,0.00,0.00}%
      \put(3246,1280){\makebox(0,0){\strut{}$0.0$}}%
      \colorrgb{0.00,0.00,0.00}%
      \put(3671,1280){\makebox(0,0){\strut{}$2.0$}}%
      \colorrgb{0.00,0.00,0.00}%
      \put(4095,1280){\makebox(0,0){\strut{}$4.0$}}%
      \colorrgb{0.00,0.00,0.00}%
      \put(4520,1280){\makebox(0,0){\strut{}$6.0$}}%
    }%
    \gplgaddtomacro\gplfronttext{%
    }%
    \gplbacktext
    \put(0,0){\includegraphics{./figures/odom_test_5_vs_6}}%
    \gplfronttext
  \end{picture}%
\endgroup

  \vspace{-2.3cm}
  \caption{\small Scan-matching as ``laser odometry": the robot moves from the
           lower left portion of the environment to the upper right, capturing
           2D range scans along its trajectory (black). The red (CSM), blue
           (NDT), and green (our method) routes show the estimated path of the
           robot derived from each method. Left figure: frequent LIDAR
           measurements.  Right figure: a downsampled version of the original
           trajectory. The proposed method's error is invariant to angular and
           locational displacement}
  \label{fig:laser_odometry}
\end{figure}

The rest of the paper is structured as follows. Section
\ref{section:definitions} defines necessary notions.  The problem of matching
panoramic 2D range scans is formulated in section \ref{section:the_problem},
and a brief review of methods matching 2D range scans is given in section
\ref{section:sota}. Section \ref{section:the_proposed_method} provides an
analysis of the proposed method. The experimental setup and results of the
proposed method against two state-of-the-art methods in common use are
illustrated in section \ref{section:results}. Section \ref{section:finale}
offers a recapitulation.
