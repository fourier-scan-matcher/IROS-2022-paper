Scan-matching is employed in robotics as a means to odometry,
primarily in non-wheeled robots where no encoders can be utilised, or as a
useful ameliorator of the ever-drifting encoder-ed odometry: scans captured at
consecutive time instances, inputted to a scan-matching algorithm, convey an
estimate as to the pose of the scan sensor at the second capture time
relative to that captured first. Scan-matching is being successfully employed
in the tasks of simultaneous localisation and mapping
\cite{am_odom_1}-\cite{am_odom_3}, local map construction
\cite{am_odom_4}-\cite{am_odom_6}, and in people-tracking systems
\cite{am_odom_7}.

cannot stress enough: the contributions must be layed out and emphasised
see. olsen 2009 for a pro

Contribution: invariant to large angular errors, pseudoinvariant in time of execution, show how to decrease orientation error from fixed number of rays, fourier for both orientation and translation --> robustness






\begin{figure}[]\centering
  \input{./figures/inner_system.tikz}
  \caption{\small FSM iteratively invokes the One-step Pose Estimation method.
           Given a pose estimate of where scan $\mathcal{S}_1$ was captured
           within $\bm{M}$, the method attempts to register $\mathcal{S}_1$ to
           $\bm{M}$ by estimating first its relative orientation and then its
           location with respect to the input pose estimate}
  \label{fig:fsm_inner}
\end{figure}
