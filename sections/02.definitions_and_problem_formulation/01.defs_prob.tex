%%%%%%%%%%%%%%%%%%%%%%%%%%%%%%%%%%%%%%%%%%%%%%%%%%%%%%%%%%%%%%%%%%%%%%%%%%%%%%%%
\begin{definition}
  \label{def:definition_1}
  \textit{Definition of a range scan captured from a conventional 2D LIDAR
  sensor.} A conventional 2D LIDAR sensor provides a finite number of ranges,
  i.e. distances to objects within its range, on a horizontal cross-section of
  its environment, at regular angular and temporal intervals, over a defined
  angular range \cite{lidar}. A range scan $\mathcal{S}$, consisting
  of $N_s$ rays over an angular range $\lambda$, is an ordered map
  $\mathcal{S} : \Theta \rightarrow \mathbb{R}_{\geq 0}$, $\Theta =
  \{\theta_n \in [-\frac{\lambda}{2}, +\frac{\lambda}{2}) : \theta_n =
  -\frac{\lambda}{2} + \lambda \frac{n}{N_s}$, $n = 0,1,\dots, N_s$$-$$1$$\}$.
  Angles $\theta_n$ are expressed relative to the sensor's heading, in the
  sensor's frame of reference. The angular distance between two consecutive
  rays is the sensor's angle increment $\gamma \triangleq \lambda/N_s$.
\end{definition}

%%%%%%%%%%%%%%%%%%%%%%%%%%%%%%%%%%%%%%%%%%%%%%%%%%%%%%%%%%%%%%%%%%%%%%%%%%%%%%%%
\begin{definition}
  \label{def:definition_3}
  \textit{Definition of a map-scan.}
  A map-scan is a virtual scan that encapsulates the same pieces of information
  as a scan derived from a physical sensor. Only their underlying operating
  principle is different due to the fact the map-scan refers to distances to
  the boundaries of a point-set, referred to as the map, rather than within a
  real environment. A map-scan is derived by means of locating intersections of
  rays emanating from the estimate of the sensor's pose estimate and the
  boundaries of the map.
\end{definition}

%%%%%%%%%%%%%%%%%%%%%%%%%%%%%%%%%%%%%%%%%%%%%%%%%%%%%%%%%%%%%%%%%%%%%%%%%%%%%%%%
\begin{problem}
  \label{prob:the_problem}
  Let a mobile robot, capable of motion in the $x-y$ plane, be equipped with a
  coplanarly mounted range scan sensor emitting $N_s$ rays. Let
  also the following be available or standing:
  \begin{itemize}
    \item The angular range of the range sensor is $360^\circ$
    \item A 2D range scan $\mathcal{S}_0$, captured at time $t_0$
    \item A 2D range scan $\mathcal{S}_1$, captured at $t_1 > t_0$
  \end{itemize}
\end{problem}
Then the objective is estimating the 3D rigid-body transformation
$\bm{T} = (\Delta x, \Delta y, \Delta \theta)$ which, when applied to the
endpoints of $\mathcal{S}_1$, aligns them to those of $\mathcal{S}_0$ with the
least error. Equivalently, roto-translation $\bm{T}$ corresponds to the
relative motion of the sensor from the pose where it captured $\mathcal{S}_0$
to the pose from which it captured $\mathcal{S}_1$.
