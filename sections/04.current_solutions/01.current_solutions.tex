Scan-matching with the use of a 2D LIDAR sensor began with IDC \cite{LuMilios},
an algorithm incorporating elements of the Iterative Closest Point (ICP)
algorithm \cite{ICP}. The latter and its variants, e.g.
\cite{weighted}-\cite{plicp}, have become the de facto scan-matching algorithms
in 2D and 3D settings, and research using ICP is still ongoing
\cite{ICP_var_1}-\cite{Marchel}. In particular, PL-ICP \cite{plicp} has been
widely adopted due to its increased accuracy among ICP variants, and the
availability of its source code. ICP and its variants, however, exhibit varying
performance \cite{icp_comp_trade}, limited by the noise level in the input
scans, the choice of prior, and the configuration of the parameters
governing their response. For these reasons, as well as for reasons of
robustness, the method of establishing correspondences shifted from
point-to-point or point-to-line to feature-to-feature. Commonly appearing
features for recognition are line segments \cite{CLS}-\cite{Wen},
SIFT features \cite{Jiayuan}, or features extracted through the use of deep
learning techniques \cite{Jiaxin}. In parallel, and for reasons of
indepencency from chance features or tailoring methods to specific
circumstances, research sprung around methods that extract or exploit
mathematical properties from range scans, or that view the problem of
scan-matching as an optimisation problem. Examples include correlation-based
methods \cite{olson},\cite{olson_2015}-\cite{Konecny}, feature distribution
matching \cite{HSM}, matching by cost function minimisation \cite{PB_PSM}, and
probabilistic methods \cite{pIC}\cite{gpm}. Among the latter, the Normal
Distributions Transform (NDT) \cite{ndt1} has gained popularity due to its
explicit modeling of measurement and pose uncertainties and its extensibility
to three dimensions \cite{ndt2}-\cite{ndt6}.

The method introduced in this paper is most akin to those of \cite{Heng} and
\cite{Jiang}. They use POMF \cite{fmt2d} in both rotation and translation
components; the latter in two dimensions and the former in one dimension. In
the latter, the requirements for a real-time solution and adequate accuracy
could not be fulfilled simultaneously. Therefore a rough output was used as an
input prior to ICP in order to overcome the problem. In the former, the
orientation error is limited by the range sensor's immutable angle increment,
but no mitigation technique is employed. Secondly, the translation component
operates in discrete space, thereby being susceptible to discretisation errors
and larger execution times as resolution increases. By contrast, the method
introduced in this paper addresses all the above issues by (a) fulfilling the
real-timeness constraint, (b) aiming at extricating the orientation error from
the sensor's angle increment, and (c) employing a continuous-space translation
method. In closing, a more detailed review of scan-matching methods may be
found in \cite{pose_selection}.
