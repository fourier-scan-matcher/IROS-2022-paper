Scan-matching with the use of a 2D LIDAR sensor began with IDC \cite{LuMilios},
an algorithm incorporating elements of the Iterative Closest Point (ICP)
algorithm \cite{ICP}. The latter and its variants, e.g.
\cite{weighted}-\cite{plicp}, have become the de facto scan-matching algorithms
in 2D and 3D settings, and research using ICP is still ongoing
\cite{ICP_var_1}-\cite{Marchel}. In particular, PL-ICP \cite{plicp} has been
widely adopted due to its increased accuracy among ICP variants, and the
availability of its source code. ICP and its variants, however, exhibit varying
performance \cite{icp_comp_trade}, limited by the noise level in the input
scans, the choice of prior, and the configuration of the parameters
governing their response. For these reasons, as well as for reasons of
robustness, the method of establishing correspondences shifted from
point-to-point or point-to-line to feature-to-feature. Commonly appearing
features for recognition are line segments \cite{CLS}\cite{Haytham}\cite{Wen},
SIFT features \cite{Jiayuan}, or features extracted through the use of deep
learning techniques \cite{Jiaxin}. In parallel, and for reasons of
indepencency from chance features or tailoring methods to specific
circumstances, research sprung around methods that extract or exploit
mathematical properties from range scans, or that view the problem of
scan-matching as an optimisation problem. Examples include correlation-based
methods \cite{olson},\cite{olson_2015}-\cite{Konecny}, feature distribution
matching \cite{HSM}, or matching by cost function minimisation \cite{PB_PSM}.

